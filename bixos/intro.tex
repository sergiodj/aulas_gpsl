\documentclass{beamer}

% colocar todos os includes comuns aqui
\usetheme{default}
\usepackage[utf8x]{inputenc}
\usepackage{ucs}
\usepackage[brazil]{babel}
\usepackage{ae,aecompl}
\usepackage[T1]{fontenc}
\usepackage{graphicx}
\usepackage{hyperref}

\setbeamertemplate{navigation symbols}{}
\hypersetup{colorlinks=true,
            pdftitle={Aula de GNU/Linux, segunda edicao},
            pdfauthor={GPSL - Grupo Pro Software Livre - UNICAMP},
            pdfsubject={Aula de GNU/Linux, segunda edicao},
            pdfkeywords={Aula GNU/Linux UNICAMP GPSL Bixos}}

\newcommand{\gpsltitle}[1]{
    \begin{frame}
        \begin{figure}[h]
            \centering
            \includegraphics[scale=0.094]{../imagens/Mooks_gpsl.png}
        \end{figure}
        \begin{center}
            \Huge #1
        \end{center}
    \end{frame}
}

% configurações do lstlisting para simular um shell

\usepackage{listings}
\usepackage{color}
\usepackage[defaultmono]{droidmono}

\lstset{
    basicstyle=\ttfamily\color{white},
    language=bash,
    stepnumber=0, % sem númvero de linhas
    %backgroundcolor=\color{black},
    tabsize=4,
    captionpos=b,
    keywordstyle=\color{green},
    showspaces=false,
    showstringspaces=false,
    inputencoding=utf8,
    extendedchars=true,
    numberstyle=\scriptsize,
    literate={á}{{\'a}}1
}

% coloca um comando de shell com a formatação correta
% o argumento opcional é o texto do prompt
\newcommand{\shellcommand}[2]{
    \textcolor{white}{#1 \ #2}
}

\newcommand{\usercmd}[2][]{
    \shellcommand{#1\$}{#2}
}

\newcommand{\rootcmd}[1]{
    \shellcommand{\#}{#1}
}

% comando para inserir um comentário com a formatação apropriada
\newcommand{\comment}[1]{
    \textcolor{gray}{\# \ {\em #1 }}
}

\newsavebox{\shellbox}
\newenvironment{shell}
{
    \global\begin{lrbox}{\shellbox}
    \begin{minipage}[c]{0.8\textwidth}
    \begin{tt}
}
{
    \nolinebreak[4]
    \end{tt}
    \end{minipage}
    \end{lrbox}
    \colorbox{black}{\usebox{\shellbox}}
}


\usepackage{xspace}

\begin{document}

\newcommand{\software}{\emph{software}\xspace}
\newcommand{\Software}{\emph{Software}\xspace}
\newcommand{\softwarelivre}{\Software Livre\xspace}
\newcommand{\opensource}{\emph{Open-Source}\xspace}
\gpsltitle{Aula Introdutória 00:\\  \softwarelivre}

\begin{frame}{Objetivos}
    \begin{itemize}
        \item Entender o que é \softwarelivre e sua importância na computação.
        \item Entender os ideais do \softwarelivre, as 4 liberdades, e as
          diferenças do simples \opensource.
        \item Conhecer os principais projetos e pessoas por trás da comunidade
          de \softwarelivre.
    \end{itemize}
\end{frame}


\begin{frame}{O que é Software livre}
    \begin{figure}[h]
        \centering
        \softwarelivre é qualquer \Software que garanta ao usuário a liberdade
        de usar, distribuir e modificar o \software, de forma a atender todas
        as suas necessidades.
    \end{figure}
\end{frame}

\begin{frame}{As 4 liberdades}
    \begin{figure}[h]
        \centering
        A definição formal de \softwarelivre foi proposta por Richard Stallman
        e se baseia em 4 liberdades, consideradas fundamentais:
        \newline
        \begin{itemize}
        \item Liberdade 0: A liberdade de executar o programa para qualquer
          propósito.
        \item Liberdade 1: A liberdade de estudar como o programa funciona e
          alterá-lo para fazê-lo funcionar como desejar.
        \item Liberdade 2: A liberdade de redistribuir cópias do programa.
        \item Liberdade 3: A liberdade de melhorar o programa e redistribuir
          para toda a comunidade.
          \end{itemize}
    \end{figure}
\end{frame}

\begin{frame}{\softwarelivre X \opensource}
   \begin{itemize}
     \item As liberdades 1, 2 e 3 exigem que o \softwarelivre seja \opensource.
        \item Todo \softwarelivre é \opensource mas, nem todo \opensource é
          \softwarelivre.
        \item Aplicações \opensource não garante a liberdade de redistribuir
          modificações, nem a manutenção das liberdades.
   \end{itemize}
\end{frame}

\begin{frame}{Projeto GNU e a \emph{Free Software Foundation}}
  %% TODO: Falar sobre ...
\end{frame}

\begin{frame}{Principais projetos}
  %% TODO: Colocar logo e informações dos principais projetos livres?
\end{frame}


\begin{frame}{Porque desenvolver \softwarelivre}
  %% Lembro que em 2012, o Sergio insistiu bastante em motivações para
  %% contribuir: melhorar skill, trabalhar na área, ser foda, etc.
  \begin{itemize}
    \item Melhorar suas habilidades como programador.
    \item Melhorar a qualidade dos seus códigos.
    \item Várias grandes empresas da área, como Red Hat, IBM e Google valorizam
      contribuições para a comunidade de \softwarelivre.
    \item Você estará contribuindo para disseminar uma cultura de cooperação, que não
      preza simplesmente o lucro, mas a disseminação do conhecimento.
  \end{itemize}
\end{frame}

\begin{frame}
  \begin{center}
    \begin{shell}
      \usercmd{ls /home/ivan}
      \usercmd{sudo su}
      \rootcmd{rm -rf \slash}
      \comment{isso não é uma boa idéia.}
      \comment{Mas vai abrir muito}
      \comment{espaço livre no seu HD}
    \end{shell}
  \end{center}
\end{frame}

\end{document}
