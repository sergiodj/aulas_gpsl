\documentclass{beamer}

% colocar todos os includes comuns aqui
\usetheme{default}
\usepackage[utf8x]{inputenc}
\usepackage{ucs}
\usepackage[brazil]{babel}
\usepackage{ae,aecompl}
\usepackage[T1]{fontenc}
\usepackage{graphicx}
\usepackage{hyperref}

\setbeamertemplate{navigation symbols}{}
\hypersetup{colorlinks=true,
            pdftitle={Aula de GNU/Linux, segunda edicao},
            pdfauthor={GPSL - Grupo Pro Software Livre - UNICAMP},
            pdfsubject={Aula de GNU/Linux, segunda edicao},
            pdfkeywords={Aula GNU/Linux UNICAMP GPSL Bixos}}

\newcommand{\gpsltitle}[1]{
    \begin{frame}
        \begin{figure}[h]
            \centering
            \includegraphics[scale=0.094]{../imagens/Mooks_gpsl.png}
        \end{figure}
        \begin{center}
            \Huge #1
        \end{center}
    \end{frame}
}

% configurações do lstlisting para simular um shell

\usepackage{listings}
\usepackage{color}
\usepackage[defaultmono]{droidmono}

\lstset{
    basicstyle=\ttfamily\color{white},
    language=bash,
    stepnumber=0, % sem númvero de linhas
    %backgroundcolor=\color{black},
    tabsize=4,
    captionpos=b,
    keywordstyle=\color{green},
    showspaces=false,
    showstringspaces=false,
    inputencoding=utf8,
    extendedchars=true,
    numberstyle=\scriptsize,
    literate={á}{{\'a}}1
}

% coloca um comando de shell com a formatação correta
% o argumento opcional é o texto do prompt
\newcommand{\shellcommand}[2]{
    \textcolor{white}{#1 \ #2}
}

\newcommand{\usercmd}[2][]{
    \shellcommand{#1\$}{#2}
}

\newcommand{\rootcmd}[1]{
    \shellcommand{\#}{#1}
}

% comando para inserir um comentário com a formatação apropriada
\newcommand{\comment}[1]{
    \textcolor{gray}{\# \ {\em #1 }}
}

\newsavebox{\shellbox}
\newenvironment{shell}
{
    \global\begin{lrbox}{\shellbox}
    \begin{minipage}[c]{0.8\textwidth}
    \begin{tt}
}
{
    \nolinebreak[4]
    \end{tt}
    \end{minipage}
    \end{lrbox}
    \colorbox{black}{\usebox{\shellbox}}
}


% Opções para o pacote hyperref
\hypersetup{colorlinks=true,
            pdftitle={Aula de GNU/Linux, segunda edicao},
            pdfauthor={GPSL - Grupo Pro Software Livre - UNICAMP},
            pdfsubject={Aula de GNU/Linux, segunda edicao},
            pdfkeywords={Aula GNU/Linux UNICAMP GPSL Bixos}}

\begin{document}

% TODO: adicionar exemplos

\gpsltitle{Aula 0: Conceitos}

\begin{frame}{Objetivos}
    \begin{itemize}
        \item Entender o processo de compilação (i.e., como o compilador
              faz a ``mágica'')
        \item Entender o processo de \textit{linkagem}, saber como
              fazer um programa básico divido em objetos
        \item Saber a diferença entre \textit{linkagem} estática e
              dinâmica
        \item Conhecer formas de obter informações sobre os binários
    \end{itemize}
\end{frame}

\begin{frame}{Processo de compilação}
    \begin{figure}[h]
        \centering
        \includegraphics[scale=0.3]{./imagens/compiler.pdf}
    \end{figure}
\end{frame}

\begin{frame}{\textit{Linkagem}}
    \begin{figure}[h]
        \centering
        \includegraphics[scale=0.4]{./imagens/linking.pdf}
    \end{figure}
\end{frame}

\begin{frame}{Loader}
    
\end{frame}

\begin{frame}{Linkagem estática/dinâmica}
    
\end{frame}

\begin{frame}{Bibliotecas}
%pkg-config, /lib, /usr/lib
\end{frame}

\begin{frame}

\begin{center}
\begin{shell}
    \usercmd{ls /home/ivan} \\
    \usercmd{sudo su} \\
    \rootcmd{rm -rf \slash} \\
    \comment{isso não é uma boa idéia}
\end{shell}
\end{center}

\end{frame}

\end{document}
